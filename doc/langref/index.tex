%THanks to M. Bravenboer:
\documentclass[10pt, b5paper,twoside,openright]{book}
\usepackage[nohead]{geometry}
\geometry{vmarginratio=2:3,hmarginratio=1:1,papersize={170mm,240mm},total={121mm,196mm}}

\usepackage{loadpackages}
\usepackage{style-tweaks}
\usepackage{color}
\usepackage{xcolor}
\usepackage{listings}
\input{commands.ltx}
%\definecolor{darkgreen}{rgb}{0,0.4,0}
\definecolor{darkblue}{rgb}{0,0,0.6}
\definecolor{darkred}{rgb}{0.6,0,0}
\definecolor{lightgray}{rgb}{0.9,0.9,0.9}
\lstdefinestyle{mystyle}
  {
  basicstyle=\ttfamily,%
  showstringspaces=false,
  keywordstyle=[1]{\color{black}\bfseries},%
  keywordstyle=[2]{\color{darkblue}\bfseries},%
  keywordstyle=[3]{\color{blue}\emph},
  keywordstyle=[4]{\color{darkred}\emph},
 	keywordstyle=[5]{\color{green}},
  stringstyle={\color{black}},%
  commentstyle={\color{darkgray}\emph},%
  emphstyle={\warn},%
  tabsize=4,
  aboveskip=10pt,
  belowskip=10pt,
  tabsize=4,
  breaklines=true,
  breakautoindent=true,
  frame=none
}
%
\lstdefinelanguage{webdsl}
  {	basicstyle=\ttfamily,
  	style=mystyle,
  	morekeywords=[1]{template, page, entity, feed, action, function},	
  	morekeywords=[2]{init,define, using, section, application, module, rule, description, note},
		morekeywords=[3]{inverse,where,order,by,asc,desc},
		morekeywords=[4]{for,select,if, in,retur,cancel,goto,var,else,is,a,as},	
		morekeywords=[5]{true,false,null},
    sensitive=false,
    morecomment=[l]{//},
    morecomment=[s]{/*}{*/},
    morestring=[b]",
    morestring=[d]’,
    literate={|[}{{\texttt{|$\!\!$[}}}1%
	    {]|}{{\texttt{]$\!\!$|}}}1%
	    {->}{$\rightarrow$ }2%
	    {=>}{$\Rightarrow$ }2,
	  numbers=right,  	    
   moredelim=[is][basicstyle]{`}{`} % to escape non-keyword words
  }
\lstnewenvironment{webdsl}[1]{\lstset{language=webdsl,caption=#1, captionpos=top}}{}
%
\lstdefinelanguage{shell}
   {
   	style=mystyle,
     morecomment=[l]{//},
     morecomment=[s]{/*}{*/},
     morestring=[b]",
     morestring=[d]’
   }
\lstnewenvironment{shell}
     {\lstset{language=shell, frame=leftline,  xleftmargin=20pt}}
     {}

%\definecolor{darkgreen}{rgb}{0,0.4,0}
\definecolor{darkblue}{rgb}{0,0,0.6}
\definecolor{darkred}{rgb}{0.6,0,0}
\definecolor{lightgray}{rgb}{0.9,0.9,0.9}
\lstdefinestyle{mystyle}
  {
  basicstyle=\ttfamily,%
  showstringspaces=false,
  keywordstyle=[1]{\color{black}\bfseries},%
  keywordstyle=[2]{\color{darkblue}\bfseries},%
  keywordstyle=[3]{\color{blue}\emph},
  keywordstyle=[4]{\color{darkred}\emph},
 	keywordstyle=[5]{\color{green}},
  stringstyle={\color{black}},%
  commentstyle={\color{darkgray}\emph},%
  emphstyle={\warn},%
  tabsize=4,
  aboveskip=10pt,
  belowskip=10pt,
  tabsize=4,
  breaklines=true,
  breakautoindent=true,
  frame=none
}
%
\lstdefinelanguage{webdsl}
  {	basicstyle=\ttfamily,
  	style=mystyle,
  	morekeywords=[1]{template, page, entity, feed, action, function},	
  	morekeywords=[2]{init,define, using, section, application, module, rule, description, note},
		morekeywords=[3]{inverse,where,order,by,asc,desc},
		morekeywords=[4]{for,select,if, in,retur,cancel,goto,var,else,is,a,as},	
		morekeywords=[5]{true,false,null},
    sensitive=false,
    morecomment=[l]{//},
    morecomment=[s]{/*}{*/},
    morestring=[b]",
    morestring=[d]’,
    literate={|[}{{\texttt{|$\!\!$[}}}1%
	    {]|}{{\texttt{]$\!\!$|}}}1%
	    {->}{$\rightarrow$ }2%
	    {=>}{$\Rightarrow$ }2,
	  numbers=right,  	    
   moredelim=[is][basicstyle]{`}{`} % to escape non-keyword words
  }
\lstnewenvironment{webdsl}[1]{\lstset{language=webdsl,caption=#1, captionpos=top}}{}
%
\lstdefinelanguage{shell}
   {
   	style=mystyle,
     morecomment=[l]{//},
     morecomment=[s]{/*}{*/},
     morestring=[b]",
     morestring=[d]’
   }
\lstnewenvironment{shell}
     {\lstset{language=shell, frame=leftline,  xleftmargin=20pt}}
     {}

\definecolor{darkgreen}{rgb}{0,0.4,0}
\definecolor{darkblue}{rgb}{0,0,0.6}
\definecolor{darkred}{rgb}{0.6,0,0}
\definecolor{lightgray}{rgb}{0.9,0.9,0.9}
\lstdefinestyle{mystyle}
  {
  basicstyle=\ttfamily,%
  showstringspaces=false,
  keywordstyle=[1]{\color{black}\bfseries},%
  keywordstyle=[2]{\color{darkblue}\bfseries},%
  keywordstyle=[3]{\color{blue}\emph},
  keywordstyle=[4]{\color{darkred}\emph},
 	keywordstyle=[5]{\color{green}},
  stringstyle={\color{black}},%
  commentstyle={\color{darkgray}\emph},%
  emphstyle={\warn},%
  tabsize=4,
  aboveskip=10pt,
  belowskip=10pt,
  tabsize=4,
  breaklines=true,
  breakautoindent=true,
  frame=none
}
%
\lstdefinelanguage{webdsl}
  {	basicstyle=\ttfamily,
  	style=mystyle,
  	morekeywords=[1]{template, page, entity, feed, action, function},	
  	morekeywords=[2]{init,define, using, section, application, module, rule, description, note},
		morekeywords=[3]{inverse,where,order,by,asc,desc},
		morekeywords=[4]{for,select,if, in,retur,cancel,goto,var,else,is,a,as},	
		morekeywords=[5]{true,false,null},
    sensitive=false,
    morecomment=[l]{//},
    morecomment=[s]{/*}{*/},
    morestring=[b]",
    morestring=[d]’,
    literate={|[}{{\texttt{|$\!\!$[}}}1%
	    {]|}{{\texttt{]$\!\!$|}}}1%
	    {->}{$\rightarrow$ }2%
	    {=>}{$\Rightarrow$ }2,
	  numbers=right,  	    
   moredelim=[is][basicstyle]{`}{`} % to escape non-keyword words
  }
\lstnewenvironment{webdsl}[1]{\lstset{language=webdsl,caption=#1, captionpos=top}}{}
%
\lstdefinelanguage{shell}
   {
   	style=mystyle,
     morecomment=[l]{//},
     morecomment=[s]{/*}{*/},
     morestring=[b]",
     morestring=[d]’
   }
\lstnewenvironment{shell}
     {\lstset{language=shell, frame=leftline,  xleftmargin=20pt}}
     {}


%end copied

\usepackage{ifpdf}
\usepackage[utf8]{inputenc}
%\usepackage[UKenglish]{babel}
\usepackage{makeidx}
\makeindex
\frenchspacing

\newcommand{\e}[1]{\emph{#1}}
\newcommand{\ii}[1]{\index{#1}}

\title{WebDSL language reference}
\author{The WebDSL Team}
\date{\today}
\ifpdf
\pdfinfo {
	/Author (The WebDSL Team)
	/Title (WebDSL language reference)
	/Subject (WebDSL)
	/Keywords ()
	/CreationDate (D:20081020135807)
}
\usepackage{graphicx}
\fi

\begin{document}
\frontmatter
\maketitle
\tableofcontents 

\mainmatter
\part{Introduction to WebDSL}
\chapter{Access Control}
% Declarative Access Control in WebDSL
% 
% Configuration of the Principal
% 
% The Access Control sublanguage is supported by a session entity that holds information regarding the currently logged in user. This session entity is called the securityContext and is configured as follows:
% 
%       access control rules
%       {
%         principal is User with credentials name, password
%       }
% 
% This states that the User entity will be the entity representing a logged in user. The credentials are not used in the current implementation (the idea is to derive a default login template). The resulting generated session entity will be:
% 
%       session securityContext
%       {
%         principal -> User
%         loggedIn :: Bool
%       }
% 
% Note that this principal declaration is used to enable the entire Access Control sublanguage.
% Authentication
% 
% Authentication must be added manually for now. Here are templates for login and logout that you can include in pages:
% 
%       entity User
%       {
%         name :: String
%         password :: Secret
%       }
% 
%       define login(){
%         var usr : User := User{};
%         form{ 
%           table{
%             row{ "Name: " input(usr.name) }
%             row{ "Password: " input(usr.password) }
%             row{ captcha() }
%             row{ action("Log In", login()) "" }
%           }
%           action login(){
%             var users : List<User> :=
%               select u from User as u 
%               where (u._name = ~usr.name);
% 
%             for (us : User in users ){
%               if (us.password.check(usr.password)){
%                 securityContext.principal := us;
%                 securityContext.loggedIn := true;
%                 return viewUser(securityContext.principal);
%               }
%             }
%             securityContext.loggedIn := false;
%             return home();
%           }
%         }
%       }
% 
%       define logout(){
%         "Logged in as " output(securityContext.principal)
%         form{
%           actionLink("Log Out", logout())
%           action logout(){
%             securityContext.loggedIn := false;
%             securityContext.principal := null;
%             return home();
%           }
%         }
%       }
% 
% When storing a secret property you need to create a digest of it:
% 
%       newUser.password := newUser.password.digest();
% 
% This makes sure the secret property is stored encrypted. A digest can be compared with an entered string using the check method:
% 
%       us.password.check(enteredpassword)
% 
% Protecting Resources
% 
% The default policy is to deny access to all pages and actions, the rules will determine what the conditions for allowing access are.
% 
% A simple rule protecting the editUser page to be only accessable by the user being edited looks like this:
% 
%       rules page editUser(u:User){
%         u = securityContext.principal     
%       }
% 
% An analysis of this rule:
% 
%     * rules: A keyword for Access Control rules
%     * page: The type of resource being protected here, all the types available in the Access Control DSL for WebDSL are: page, action, template, function. The rules on pages protect the viewing of pages, action rules protect the execution of actions, template rules determine whether a template is visible in the including page, and finally rules on functions are lifted to the action invoking the function.
%     * editUser: The name of the resource the rule will apply to.
%     * (u:User): The arguments (if any) of the resource, the types of the arguments are used when matching. This also specifies what variables can be used in the checks.
%     * u = securityContext.principal: the check that determines whether access to this resource is allowed, this check is typechecked to be a correct boolean expression. The use of securityContext.principal implies that the securityContext is not null and the user is logged in (these extra checks are generated automatically).
% 
% Matching can be done a bit more freely using a trailing * as wildcard character, both in resource name and arguments:
% 
%     rules page viewUs*(*){
%       true
%     }
% 
% When more fine-grained control is needed for rules, it is possible to specify nested rules. This implies that the nested rule is only valid for usage of that resource inside the parent resource. The allowed combinations are page - action, template - action, page - template. The next example shows nested rules for actions in a page:
% 
%       rules page editDocument(d:Document){
%         d.author = securityContext.principal
%         rules action save(){
%           d.author = securityContext.principal
%         }
%         rules action cancel(){
%           d.author = securityContext.principal
%         }      
%       }
% 
% This flexibility is often not necessary, and it is also inconvenient having to explicitly allow all the actions on the page, for these reasons some extra desugaring rules were added. When specifying a check on a page or template without nested checks, a generic action rules block with the same check is added to it by default. For example:
% 
%       rules page editDocument(d:Document){
%         d.author = securityContext.principal     
%       }
% 
% becomes
% 
%       rules page editDocument(d:Document){
%         d.author = securityContext.principal
%         rules action *(*)
%         {
%           d.author = securityContext.principal
%         }    
%       }
% 
% Reuse in Access Control rules
% 
% Predicates are functions consisting of one boolean expression, which allows reusing complicated expressions, or simply giving better structure to the policy implementation. An example of a predicate:
% 
%       predicate mayViewDocument (u:User, d:Document){
%         d.author = securityContext.principal
%         || u in d.allowedUsers
%       }
%       rules page viewDocument(d:Document){
%         mayViewDocument(securityContext.principal,d)
%       }
%       rules page showDocument(d:Document){
%         mayViewDocument(securityContext.principal,d)
%       }
% 
% Pointcuts are groups of resources to which conditions can be specified at once. Especially the open parts of the web application are easy to handle with pointcuts, an example:
% 
%       pointcut openSections(){
%         page home(),
%         page createDocument(),
%         page createUser(),
%         page viewUser(*)
%       }
%       rules pointcut openSections(){
%         true
%       }
% 
% Pointcuts can also be used with parameters:
% 
%       pointcut ownUserSections(u:User){
%         page showUser(u),
%         page viewUser(u),
%         page someUserTask(u,*)
%       }
%       rules pointcut ownUserSections(u:User){
%         u = securityContext.principal
%       }
% 
% Inferring Visibility
% 
% A disabled page or action redirects to a very simple page stating access denied. Since this is not very user friendly, the visibility of navigate links and action buttons/links are automatically made conditional using the same check as the corresponding resource. An example conditional navigate:
% 
%     if(mayViewDocument(securityContext.principal,d)){
%       navigate(viewDocument(d)){ "view " output(d.title) } 
%     }
% 
% When using conditional forms it is often more convenient to put the form in a template, and control the visibility by a rule on the template.
% Using Entities
% 
% Access Control policies that rely on extra data can create new or extend existing properties. An example of extending an entity is adding a set of users property to a document representing the users allowed access to that document:
% 
%       extend entity Document{
%         allowedUsers -> Set<User>
%       }
% 
% Administration of Access Control
% 
% Administration of Access Control in WebDSL is done by the normal WebDSL page definitions. All the data of the Access Control policy is integrated into the WebDSL application. An option is to incorporate the administration into an existing page with a template. This example illustrates the use of a template for administration:
% 
%       define allowedUsersRow(document:Document){
%         row{ "Allowed Users:" input(document.allowedUsers) }
%       }
% 
% The template call for this template is added to the editDocument page:
% 
%       table{
%         row{ "Title:" input(document.title) }
%         row{ "Text:" input(document.text) }
%         row{ "Author:" input(document.author) }  
%         allowedUsersRow(document)
%       }
% 
% By using a template the Access Control can be disabled easily by not including the access control definitions and the template. The unresolved template definitions will give a warning but the page will generate normally and ignore the template call.



\chapter{Styling}
Styling sublanguage

This is a normal paragraph.

This is a preformatted
code block.

This is code: a := b * (c + d);

    This is verbatim: a := b * (c + d);

emph strong
Overview of styling properties and their types:
Property 	Type 	Element 	Value 	Example
align 	Align 		

   1. left
   2. right
   3. center

	align := Align.center;
background-color 	Color 		

    * #hexcolor
    * e.g. white, blue, ...

	background-color := Color.red
border-color 	Color 		see background-color 	border-color := #f0f0f0;
border-style 	BorderStyle 		

   1. solid
   2. dotted
   3. dashed
   4. double
   5. none

	border-style := BorderStyle.solid;
border-width 	Length 			border-width := 1px;
font 	Font 		e.g. Lucida, Arial, Verdana, e.g. 	font := Font.Lucida;
font-color 	Color 		see background-color 	font-color := #f4f4f4;
font-line 	Line 		

   1. under
   2. over
   3. through
   4. none

	font-line := Line.under;
font-size 	Length 		e.g. 2px, 1em 	font-size := 1em;
font-style 	FontStyle 		

   1. italic
   2. bold
   3. normal

	font-style := FontStyle.italic;


\chapter{WebWorkflow}

\chapter{Ajax}


\part{Core Language Reference}
\chapter{Access Control}
% Declarative Access Control in WebDSL
% 
% Configuration of the Principal
% 
% The Access Control sublanguage is supported by a session entity that holds information regarding the currently logged in user. This session entity is called the securityContext and is configured as follows:
% 
%       access control rules
%       {
%         principal is User with credentials name, password
%       }
% 
% This states that the User entity will be the entity representing a logged in user. The credentials are not used in the current implementation (the idea is to derive a default login template). The resulting generated session entity will be:
% 
%       session securityContext
%       {
%         principal -> User
%         loggedIn :: Bool
%       }
% 
% Note that this principal declaration is used to enable the entire Access Control sublanguage.
% Authentication
% 
% Authentication must be added manually for now. Here are templates for login and logout that you can include in pages:
% 
%       entity User
%       {
%         name :: String
%         password :: Secret
%       }
% 
%       define login(){
%         var usr : User := User{};
%         form{ 
%           table{
%             row{ "Name: " input(usr.name) }
%             row{ "Password: " input(usr.password) }
%             row{ captcha() }
%             row{ action("Log In", login()) "" }
%           }
%           action login(){
%             var users : List<User> :=
%               select u from User as u 
%               where (u._name = ~usr.name);
% 
%             for (us : User in users ){
%               if (us.password.check(usr.password)){
%                 securityContext.principal := us;
%                 securityContext.loggedIn := true;
%                 return viewUser(securityContext.principal);
%               }
%             }
%             securityContext.loggedIn := false;
%             return home();
%           }
%         }
%       }
% 
%       define logout(){
%         "Logged in as " output(securityContext.principal)
%         form{
%           actionLink("Log Out", logout())
%           action logout(){
%             securityContext.loggedIn := false;
%             securityContext.principal := null;
%             return home();
%           }
%         }
%       }
% 
% When storing a secret property you need to create a digest of it:
% 
%       newUser.password := newUser.password.digest();
% 
% This makes sure the secret property is stored encrypted. A digest can be compared with an entered string using the check method:
% 
%       us.password.check(enteredpassword)
% 
% Protecting Resources
% 
% The default policy is to deny access to all pages and actions, the rules will determine what the conditions for allowing access are.
% 
% A simple rule protecting the editUser page to be only accessable by the user being edited looks like this:
% 
%       rules page editUser(u:User){
%         u = securityContext.principal     
%       }
% 
% An analysis of this rule:
% 
%     * rules: A keyword for Access Control rules
%     * page: The type of resource being protected here, all the types available in the Access Control DSL for WebDSL are: page, action, template, function. The rules on pages protect the viewing of pages, action rules protect the execution of actions, template rules determine whether a template is visible in the including page, and finally rules on functions are lifted to the action invoking the function.
%     * editUser: The name of the resource the rule will apply to.
%     * (u:User): The arguments (if any) of the resource, the types of the arguments are used when matching. This also specifies what variables can be used in the checks.
%     * u = securityContext.principal: the check that determines whether access to this resource is allowed, this check is typechecked to be a correct boolean expression. The use of securityContext.principal implies that the securityContext is not null and the user is logged in (these extra checks are generated automatically).
% 
% Matching can be done a bit more freely using a trailing * as wildcard character, both in resource name and arguments:
% 
%     rules page viewUs*(*){
%       true
%     }
% 
% When more fine-grained control is needed for rules, it is possible to specify nested rules. This implies that the nested rule is only valid for usage of that resource inside the parent resource. The allowed combinations are page - action, template - action, page - template. The next example shows nested rules for actions in a page:
% 
%       rules page editDocument(d:Document){
%         d.author = securityContext.principal
%         rules action save(){
%           d.author = securityContext.principal
%         }
%         rules action cancel(){
%           d.author = securityContext.principal
%         }      
%       }
% 
% This flexibility is often not necessary, and it is also inconvenient having to explicitly allow all the actions on the page, for these reasons some extra desugaring rules were added. When specifying a check on a page or template without nested checks, a generic action rules block with the same check is added to it by default. For example:
% 
%       rules page editDocument(d:Document){
%         d.author = securityContext.principal     
%       }
% 
% becomes
% 
%       rules page editDocument(d:Document){
%         d.author = securityContext.principal
%         rules action *(*)
%         {
%           d.author = securityContext.principal
%         }    
%       }
% 
% Reuse in Access Control rules
% 
% Predicates are functions consisting of one boolean expression, which allows reusing complicated expressions, or simply giving better structure to the policy implementation. An example of a predicate:
% 
%       predicate mayViewDocument (u:User, d:Document){
%         d.author = securityContext.principal
%         || u in d.allowedUsers
%       }
%       rules page viewDocument(d:Document){
%         mayViewDocument(securityContext.principal,d)
%       }
%       rules page showDocument(d:Document){
%         mayViewDocument(securityContext.principal,d)
%       }
% 
% Pointcuts are groups of resources to which conditions can be specified at once. Especially the open parts of the web application are easy to handle with pointcuts, an example:
% 
%       pointcut openSections(){
%         page home(),
%         page createDocument(),
%         page createUser(),
%         page viewUser(*)
%       }
%       rules pointcut openSections(){
%         true
%       }
% 
% Pointcuts can also be used with parameters:
% 
%       pointcut ownUserSections(u:User){
%         page showUser(u),
%         page viewUser(u),
%         page someUserTask(u,*)
%       }
%       rules pointcut ownUserSections(u:User){
%         u = securityContext.principal
%       }
% 
% Inferring Visibility
% 
% A disabled page or action redirects to a very simple page stating access denied. Since this is not very user friendly, the visibility of navigate links and action buttons/links are automatically made conditional using the same check as the corresponding resource. An example conditional navigate:
% 
%     if(mayViewDocument(securityContext.principal,d)){
%       navigate(viewDocument(d)){ "view " output(d.title) } 
%     }
% 
% When using conditional forms it is often more convenient to put the form in a template, and control the visibility by a rule on the template.
% Using Entities
% 
% Access Control policies that rely on extra data can create new or extend existing properties. An example of extending an entity is adding a set of users property to a document representing the users allowed access to that document:
% 
%       extend entity Document{
%         allowedUsers -> Set<User>
%       }
% 
% Administration of Access Control
% 
% Administration of Access Control in WebDSL is done by the normal WebDSL page definitions. All the data of the Access Control policy is integrated into the WebDSL application. An option is to incorporate the administration into an existing page with a template. This example illustrates the use of a template for administration:
% 
%       define allowedUsersRow(document:Document){
%         row{ "Allowed Users:" input(document.allowedUsers) }
%       }
% 
% The template call for this template is added to the editDocument page:
% 
%       table{
%         row{ "Title:" input(document.title) }
%         row{ "Text:" input(document.text) }
%         row{ "Author:" input(document.author) }  
%         allowedUsersRow(document)
%       }
% 
% By using a template the Access Control can be disabled easily by not including the access control definitions and the template. The unresolved template definitions will give a warning but the page will generate normally and ignore the template call.



\chapter{Styling}
Styling sublanguage

This is a normal paragraph.

This is a preformatted
code block.

This is code: a := b * (c + d);

    This is verbatim: a := b * (c + d);

emph strong
Overview of styling properties and their types:
Property 	Type 	Element 	Value 	Example
align 	Align 		

   1. left
   2. right
   3. center

	align := Align.center;
background-color 	Color 		

    * #hexcolor
    * e.g. white, blue, ...

	background-color := Color.red
border-color 	Color 		see background-color 	border-color := #f0f0f0;
border-style 	BorderStyle 		

   1. solid
   2. dotted
   3. dashed
   4. double
   5. none

	border-style := BorderStyle.solid;
border-width 	Length 			border-width := 1px;
font 	Font 		e.g. Lucida, Arial, Verdana, e.g. 	font := Font.Lucida;
font-color 	Color 		see background-color 	font-color := #f4f4f4;
font-line 	Line 		

   1. under
   2. over
   3. through
   4. none

	font-line := Line.under;
font-size 	Length 		e.g. 2px, 1em 	font-size := 1em;
font-style 	FontStyle 		

   1. italic
   2. bold
   3. normal

	font-style := FontStyle.italic;


\chapter{WebWorkflow}

\chapter{Ajax}


\part{Module Reference}
\chapter{Access Control}
% Declarative Access Control in WebDSL
% 
% Configuration of the Principal
% 
% The Access Control sublanguage is supported by a session entity that holds information regarding the currently logged in user. This session entity is called the securityContext and is configured as follows:
% 
%       access control rules
%       {
%         principal is User with credentials name, password
%       }
% 
% This states that the User entity will be the entity representing a logged in user. The credentials are not used in the current implementation (the idea is to derive a default login template). The resulting generated session entity will be:
% 
%       session securityContext
%       {
%         principal -> User
%         loggedIn :: Bool
%       }
% 
% Note that this principal declaration is used to enable the entire Access Control sublanguage.
% Authentication
% 
% Authentication must be added manually for now. Here are templates for login and logout that you can include in pages:
% 
%       entity User
%       {
%         name :: String
%         password :: Secret
%       }
% 
%       define login(){
%         var usr : User := User{};
%         form{ 
%           table{
%             row{ "Name: " input(usr.name) }
%             row{ "Password: " input(usr.password) }
%             row{ captcha() }
%             row{ action("Log In", login()) "" }
%           }
%           action login(){
%             var users : List<User> :=
%               select u from User as u 
%               where (u._name = ~usr.name);
% 
%             for (us : User in users ){
%               if (us.password.check(usr.password)){
%                 securityContext.principal := us;
%                 securityContext.loggedIn := true;
%                 return viewUser(securityContext.principal);
%               }
%             }
%             securityContext.loggedIn := false;
%             return home();
%           }
%         }
%       }
% 
%       define logout(){
%         "Logged in as " output(securityContext.principal)
%         form{
%           actionLink("Log Out", logout())
%           action logout(){
%             securityContext.loggedIn := false;
%             securityContext.principal := null;
%             return home();
%           }
%         }
%       }
% 
% When storing a secret property you need to create a digest of it:
% 
%       newUser.password := newUser.password.digest();
% 
% This makes sure the secret property is stored encrypted. A digest can be compared with an entered string using the check method:
% 
%       us.password.check(enteredpassword)
% 
% Protecting Resources
% 
% The default policy is to deny access to all pages and actions, the rules will determine what the conditions for allowing access are.
% 
% A simple rule protecting the editUser page to be only accessable by the user being edited looks like this:
% 
%       rules page editUser(u:User){
%         u = securityContext.principal     
%       }
% 
% An analysis of this rule:
% 
%     * rules: A keyword for Access Control rules
%     * page: The type of resource being protected here, all the types available in the Access Control DSL for WebDSL are: page, action, template, function. The rules on pages protect the viewing of pages, action rules protect the execution of actions, template rules determine whether a template is visible in the including page, and finally rules on functions are lifted to the action invoking the function.
%     * editUser: The name of the resource the rule will apply to.
%     * (u:User): The arguments (if any) of the resource, the types of the arguments are used when matching. This also specifies what variables can be used in the checks.
%     * u = securityContext.principal: the check that determines whether access to this resource is allowed, this check is typechecked to be a correct boolean expression. The use of securityContext.principal implies that the securityContext is not null and the user is logged in (these extra checks are generated automatically).
% 
% Matching can be done a bit more freely using a trailing * as wildcard character, both in resource name and arguments:
% 
%     rules page viewUs*(*){
%       true
%     }
% 
% When more fine-grained control is needed for rules, it is possible to specify nested rules. This implies that the nested rule is only valid for usage of that resource inside the parent resource. The allowed combinations are page - action, template - action, page - template. The next example shows nested rules for actions in a page:
% 
%       rules page editDocument(d:Document){
%         d.author = securityContext.principal
%         rules action save(){
%           d.author = securityContext.principal
%         }
%         rules action cancel(){
%           d.author = securityContext.principal
%         }      
%       }
% 
% This flexibility is often not necessary, and it is also inconvenient having to explicitly allow all the actions on the page, for these reasons some extra desugaring rules were added. When specifying a check on a page or template without nested checks, a generic action rules block with the same check is added to it by default. For example:
% 
%       rules page editDocument(d:Document){
%         d.author = securityContext.principal     
%       }
% 
% becomes
% 
%       rules page editDocument(d:Document){
%         d.author = securityContext.principal
%         rules action *(*)
%         {
%           d.author = securityContext.principal
%         }    
%       }
% 
% Reuse in Access Control rules
% 
% Predicates are functions consisting of one boolean expression, which allows reusing complicated expressions, or simply giving better structure to the policy implementation. An example of a predicate:
% 
%       predicate mayViewDocument (u:User, d:Document){
%         d.author = securityContext.principal
%         || u in d.allowedUsers
%       }
%       rules page viewDocument(d:Document){
%         mayViewDocument(securityContext.principal,d)
%       }
%       rules page showDocument(d:Document){
%         mayViewDocument(securityContext.principal,d)
%       }
% 
% Pointcuts are groups of resources to which conditions can be specified at once. Especially the open parts of the web application are easy to handle with pointcuts, an example:
% 
%       pointcut openSections(){
%         page home(),
%         page createDocument(),
%         page createUser(),
%         page viewUser(*)
%       }
%       rules pointcut openSections(){
%         true
%       }
% 
% Pointcuts can also be used with parameters:
% 
%       pointcut ownUserSections(u:User){
%         page showUser(u),
%         page viewUser(u),
%         page someUserTask(u,*)
%       }
%       rules pointcut ownUserSections(u:User){
%         u = securityContext.principal
%       }
% 
% Inferring Visibility
% 
% A disabled page or action redirects to a very simple page stating access denied. Since this is not very user friendly, the visibility of navigate links and action buttons/links are automatically made conditional using the same check as the corresponding resource. An example conditional navigate:
% 
%     if(mayViewDocument(securityContext.principal,d)){
%       navigate(viewDocument(d)){ "view " output(d.title) } 
%     }
% 
% When using conditional forms it is often more convenient to put the form in a template, and control the visibility by a rule on the template.
% Using Entities
% 
% Access Control policies that rely on extra data can create new or extend existing properties. An example of extending an entity is adding a set of users property to a document representing the users allowed access to that document:
% 
%       extend entity Document{
%         allowedUsers -> Set<User>
%       }
% 
% Administration of Access Control
% 
% Administration of Access Control in WebDSL is done by the normal WebDSL page definitions. All the data of the Access Control policy is integrated into the WebDSL application. An option is to incorporate the administration into an existing page with a template. This example illustrates the use of a template for administration:
% 
%       define allowedUsersRow(document:Document){
%         row{ "Allowed Users:" input(document.allowedUsers) }
%       }
% 
% The template call for this template is added to the editDocument page:
% 
%       table{
%         row{ "Title:" input(document.title) }
%         row{ "Text:" input(document.text) }
%         row{ "Author:" input(document.author) }  
%         allowedUsersRow(document)
%       }
% 
% By using a template the Access Control can be disabled easily by not including the access control definitions and the template. The unresolved template definitions will give a warning but the page will generate normally and ignore the template call.



\chapter{Styling}
Styling sublanguage

This is a normal paragraph.

This is a preformatted
code block.

This is code: a := b * (c + d);

    This is verbatim: a := b * (c + d);

emph strong
Overview of styling properties and their types:
Property 	Type 	Element 	Value 	Example
align 	Align 		

   1. left
   2. right
   3. center

	align := Align.center;
background-color 	Color 		

    * #hexcolor
    * e.g. white, blue, ...

	background-color := Color.red
border-color 	Color 		see background-color 	border-color := #f0f0f0;
border-style 	BorderStyle 		

   1. solid
   2. dotted
   3. dashed
   4. double
   5. none

	border-style := BorderStyle.solid;
border-width 	Length 			border-width := 1px;
font 	Font 		e.g. Lucida, Arial, Verdana, e.g. 	font := Font.Lucida;
font-color 	Color 		see background-color 	font-color := #f4f4f4;
font-line 	Line 		

   1. under
   2. over
   3. through
   4. none

	font-line := Line.under;
font-size 	Length 		e.g. 2px, 1em 	font-size := 1em;
font-style 	FontStyle 		

   1. italic
   2. bold
   3. normal

	font-style := FontStyle.italic;


\chapter{WebWorkflow}

\chapter{Ajax}



\backmatter
\appendix
\part{Examples}

\part{A note about Stratego/XT}
\chapter{Access Control}
% Declarative Access Control in WebDSL
% 
% Configuration of the Principal
% 
% The Access Control sublanguage is supported by a session entity that holds information regarding the currently logged in user. This session entity is called the securityContext and is configured as follows:
% 
%       access control rules
%       {
%         principal is User with credentials name, password
%       }
% 
% This states that the User entity will be the entity representing a logged in user. The credentials are not used in the current implementation (the idea is to derive a default login template). The resulting generated session entity will be:
% 
%       session securityContext
%       {
%         principal -> User
%         loggedIn :: Bool
%       }
% 
% Note that this principal declaration is used to enable the entire Access Control sublanguage.
% Authentication
% 
% Authentication must be added manually for now. Here are templates for login and logout that you can include in pages:
% 
%       entity User
%       {
%         name :: String
%         password :: Secret
%       }
% 
%       define login(){
%         var usr : User := User{};
%         form{ 
%           table{
%             row{ "Name: " input(usr.name) }
%             row{ "Password: " input(usr.password) }
%             row{ captcha() }
%             row{ action("Log In", login()) "" }
%           }
%           action login(){
%             var users : List<User> :=
%               select u from User as u 
%               where (u._name = ~usr.name);
% 
%             for (us : User in users ){
%               if (us.password.check(usr.password)){
%                 securityContext.principal := us;
%                 securityContext.loggedIn := true;
%                 return viewUser(securityContext.principal);
%               }
%             }
%             securityContext.loggedIn := false;
%             return home();
%           }
%         }
%       }
% 
%       define logout(){
%         "Logged in as " output(securityContext.principal)
%         form{
%           actionLink("Log Out", logout())
%           action logout(){
%             securityContext.loggedIn := false;
%             securityContext.principal := null;
%             return home();
%           }
%         }
%       }
% 
% When storing a secret property you need to create a digest of it:
% 
%       newUser.password := newUser.password.digest();
% 
% This makes sure the secret property is stored encrypted. A digest can be compared with an entered string using the check method:
% 
%       us.password.check(enteredpassword)
% 
% Protecting Resources
% 
% The default policy is to deny access to all pages and actions, the rules will determine what the conditions for allowing access are.
% 
% A simple rule protecting the editUser page to be only accessable by the user being edited looks like this:
% 
%       rules page editUser(u:User){
%         u = securityContext.principal     
%       }
% 
% An analysis of this rule:
% 
%     * rules: A keyword for Access Control rules
%     * page: The type of resource being protected here, all the types available in the Access Control DSL for WebDSL are: page, action, template, function. The rules on pages protect the viewing of pages, action rules protect the execution of actions, template rules determine whether a template is visible in the including page, and finally rules on functions are lifted to the action invoking the function.
%     * editUser: The name of the resource the rule will apply to.
%     * (u:User): The arguments (if any) of the resource, the types of the arguments are used when matching. This also specifies what variables can be used in the checks.
%     * u = securityContext.principal: the check that determines whether access to this resource is allowed, this check is typechecked to be a correct boolean expression. The use of securityContext.principal implies that the securityContext is not null and the user is logged in (these extra checks are generated automatically).
% 
% Matching can be done a bit more freely using a trailing * as wildcard character, both in resource name and arguments:
% 
%     rules page viewUs*(*){
%       true
%     }
% 
% When more fine-grained control is needed for rules, it is possible to specify nested rules. This implies that the nested rule is only valid for usage of that resource inside the parent resource. The allowed combinations are page - action, template - action, page - template. The next example shows nested rules for actions in a page:
% 
%       rules page editDocument(d:Document){
%         d.author = securityContext.principal
%         rules action save(){
%           d.author = securityContext.principal
%         }
%         rules action cancel(){
%           d.author = securityContext.principal
%         }      
%       }
% 
% This flexibility is often not necessary, and it is also inconvenient having to explicitly allow all the actions on the page, for these reasons some extra desugaring rules were added. When specifying a check on a page or template without nested checks, a generic action rules block with the same check is added to it by default. For example:
% 
%       rules page editDocument(d:Document){
%         d.author = securityContext.principal     
%       }
% 
% becomes
% 
%       rules page editDocument(d:Document){
%         d.author = securityContext.principal
%         rules action *(*)
%         {
%           d.author = securityContext.principal
%         }    
%       }
% 
% Reuse in Access Control rules
% 
% Predicates are functions consisting of one boolean expression, which allows reusing complicated expressions, or simply giving better structure to the policy implementation. An example of a predicate:
% 
%       predicate mayViewDocument (u:User, d:Document){
%         d.author = securityContext.principal
%         || u in d.allowedUsers
%       }
%       rules page viewDocument(d:Document){
%         mayViewDocument(securityContext.principal,d)
%       }
%       rules page showDocument(d:Document){
%         mayViewDocument(securityContext.principal,d)
%       }
% 
% Pointcuts are groups of resources to which conditions can be specified at once. Especially the open parts of the web application are easy to handle with pointcuts, an example:
% 
%       pointcut openSections(){
%         page home(),
%         page createDocument(),
%         page createUser(),
%         page viewUser(*)
%       }
%       rules pointcut openSections(){
%         true
%       }
% 
% Pointcuts can also be used with parameters:
% 
%       pointcut ownUserSections(u:User){
%         page showUser(u),
%         page viewUser(u),
%         page someUserTask(u,*)
%       }
%       rules pointcut ownUserSections(u:User){
%         u = securityContext.principal
%       }
% 
% Inferring Visibility
% 
% A disabled page or action redirects to a very simple page stating access denied. Since this is not very user friendly, the visibility of navigate links and action buttons/links are automatically made conditional using the same check as the corresponding resource. An example conditional navigate:
% 
%     if(mayViewDocument(securityContext.principal,d)){
%       navigate(viewDocument(d)){ "view " output(d.title) } 
%     }
% 
% When using conditional forms it is often more convenient to put the form in a template, and control the visibility by a rule on the template.
% Using Entities
% 
% Access Control policies that rely on extra data can create new or extend existing properties. An example of extending an entity is adding a set of users property to a document representing the users allowed access to that document:
% 
%       extend entity Document{
%         allowedUsers -> Set<User>
%       }
% 
% Administration of Access Control
% 
% Administration of Access Control in WebDSL is done by the normal WebDSL page definitions. All the data of the Access Control policy is integrated into the WebDSL application. An option is to incorporate the administration into an existing page with a template. This example illustrates the use of a template for administration:
% 
%       define allowedUsersRow(document:Document){
%         row{ "Allowed Users:" input(document.allowedUsers) }
%       }
% 
% The template call for this template is added to the editDocument page:
% 
%       table{
%         row{ "Title:" input(document.title) }
%         row{ "Text:" input(document.text) }
%         row{ "Author:" input(document.author) }  
%         allowedUsersRow(document)
%       }
% 
% By using a template the Access Control can be disabled easily by not including the access control definitions and the template. The unresolved template definitions will give a warning but the page will generate normally and ignore the template call.



\chapter{Styling}
Styling sublanguage

This is a normal paragraph.

This is a preformatted
code block.

This is code: a := b * (c + d);

    This is verbatim: a := b * (c + d);

emph strong
Overview of styling properties and their types:
Property 	Type 	Element 	Value 	Example
align 	Align 		

   1. left
   2. right
   3. center

	align := Align.center;
background-color 	Color 		

    * #hexcolor
    * e.g. white, blue, ...

	background-color := Color.red
border-color 	Color 		see background-color 	border-color := #f0f0f0;
border-style 	BorderStyle 		

   1. solid
   2. dotted
   3. dashed
   4. double
   5. none

	border-style := BorderStyle.solid;
border-width 	Length 			border-width := 1px;
font 	Font 		e.g. Lucida, Arial, Verdana, e.g. 	font := Font.Lucida;
font-color 	Color 		see background-color 	font-color := #f4f4f4;
font-line 	Line 		

   1. under
   2. over
   3. through
   4. none

	font-line := Line.under;
font-size 	Length 		e.g. 2px, 1em 	font-size := 1em;
font-style 	FontStyle 		

   1. italic
   2. bold
   3. normal

	font-style := FontStyle.italic;


\chapter{WebWorkflow}

\chapter{Ajax}


\part{Architecture of the WebDSL compiler}

\part{Debugging guide to WebDSL applications and the compiler}

\part{Index}
\lstlistoflistings
\printindex
\end{document}
